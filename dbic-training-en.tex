\usepackage[utf8]{inputenc}
\usepackage[T1]{fontenc}
\usepackage{mathptmx}
\usepackage[scaled=.90]{helvet}
\usepackage{courier}
\usepackage{caption}
\captionsetup{labelformat=empty,labelsep=none}
\usepackage{verbatim}
\usepackage{hyperref}
\usepackage{listings}
\usepackage{ulem}
\lstset{language=Perl,basicstyle=\normalsize,tabsize=3,showstringspaces=false}

\title{DBIx::Class Training}
\author{Stefan Hornburg (Racke), Peter Mottram (SysPete)}
\date{Perl Dancer Conference 2015, Vienna, 20th October 2015}

\begin{document}
\maketitle{}

\begin{frame}
  \titlepage
\end{frame}

\tableofcontents

\section{Introduction}

\begin{frame}{SQL is ...}
\begin{itemize}
\item SQL is ... boring
\item SQL is ... complex
\item SQL is ... incompatible
\end{itemize}
\end{frame}

\section{Schema Classes}

\begin{frame}{Schema Classes}
\begin{itemize}
\item Country
\item User
\end{itemize}
\end{frame}

\subsection{Country Result Class}

\begin{frame}{Country / Vanilla DBIx::Class}

\end{frame}

\begin{frame}{Country / Candy DBIx::Class}

\end{frame}

\subsection{User Result Class}

\begin{frame}{User / Vanilla DBIx::Class}

\end{frame}

\begin{frame}{User / Candy DBIx::Class}

\end{frame}

\section{Basic and Advanced Queries}
\begin{frame}{Basic and Advanced Queries}
\end{frame}

\subsection{Correlated Subqueries}
\begin{frame}{Correlated Subqueries}
\end{frame}

\section{Using relationships}
\begin{frame}{Using relationships}
\end{frame}

\section{Extending Schema}
\begin{frame}{Extending Schema}
\end{frame}

\subsection{Helpers}
\subsubsection{ResultSet Helpers}
\begin{description}
\item[Me] 
simple exact way to set table aliases correctly in queries
\item[Random] small random subset
great for related product searches where you want a small 
random subset of the results
\item[CorrelateRelationship] 
makes correlated subqueries really simple
\end{description}

\begin{frame}{ResultSet Helpers}
\begin{description}
\item[Me] correct table aliases
\item[Random] small random subset
\item[CorrelateRelationship]  
\end{description}
\end{frame}

\subsubsection{Shortcuts}
\begin{description}
\item[AddColumns] 
\item[Columns]
\item[Distinct]
\item[GroupBy] 
\item[HRI]
uses this all the time
\item[HasRows] much faster than count for very large resultsets
\item[Limit]
\item[OrderBy]
\item[Page]
\item[Prefetch]
\item[Rows]
\item[Search::{Not}Like]
\item[Search::{Not}Null] 
\end{description}

\subsubsection{Schema}
\begin{description}
\item[DateTime]
much simpler query construction for DateTime inflated fields
\item[QuoteNames] 
forces quote\_names even if someone misses it in their 
config yml - everyone should use this
\end{description}

\begin{frame}{Schema Helpers}
\begin{description}
\item[DateTime] DateTime inflated fields
\item[QuoteNames] forces quote\_names
\end{description}
\end{frame}

\section{Writing Tests}
\begin{frame}{Writing Tests}
\end{frame}

\section{Deployment Handler}
\begin{frame}{Deployment Handler}
\end{frame}

\end{document}

%%% Local Variables: 
%%% mode: latex
%%% TeX-master: t
%%% End: 
